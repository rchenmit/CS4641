\documentclass[12pt]{article}
\usepackage{geometry}
\usepackage{amsmath}
\usepackage{amssymb}
\usepackage{mathtools}
\usepackage{enumitem}
\usepackage{fancyhdr}
\usepackage{tikz}
\usetikzlibrary{trees}
\pagestyle{fancy}

\lhead{CS 4641 lecture}
\chead{Game Theory Day 3}
\rhead{4/22/2014}

\begin{document}
%\begin{enumerate}
%\end{enumerate}

\title{20140422-Game Theory 3}

\section{Game Theory 3 - Goat game}

\subsection{Goat game description}You have $n$ farmers; you have $36$ goats.  Every farmer can bring between 0 and 36 goats inclusive, $[0, 36]$. The number of goats that you can bring, is a strategy. 

We assume that $D = \sqrt(36-G)$, where D is the number of dollars you get, and $G$ is the number of goats that you bring. 

We represent the outcome, the number of goats each person brings as:

$(g_1, g_2, ... g_n)$.

Let's presuppose that there is a NE outcome:

$NE (g_1^\star, g_2^\star, ... g_n^\star)$, 

where:

$g_i^\star = \underset{g_i}{\operatorname{argmax}} [g_i \sqrt(36-g_i-G_i^\star)]$

Now, we want to know what will maximize the above equation for $g_i^\star$. To know the optimal value, we TAKE THE DERIVATIVE and SET IT EQUAL TO ZERO:

$0 = \frac{\delta}{\delta g} g_i \sqrt(36-g_i-G_{-i}^\star)$

$g_i^\star = \frac{2}{3}(36-G_{-i}^\star) = 24-\frac{2}{3} G_{-i}^\star$
\\
therefore, we set:
$g^\star = g_1^\star = g_2^\star = ... = g_n^\star$

$g^\star = 24 - \frac{2}{3} (n-1) g^\star$

$= \frac{72}{2n+1}$

Therefore you end up with 

$36-\frac{36}{2n+1}$

\subsubsection{example}

If you just want to maximiaze the \textbf{original equations}: 

Left to their own , devices, this is what will happen if there are $n=24$ goat herders $\rightarrow$ then you end up with 1.26 cents per farmer

But, the NE, if there are $n=24$ goat herders $\rightarrow$ then you end up with 3.46 cents per farmer. 

Note: the OPTIMAL answer is NOT the same as the NE. This is called the \textbf{tragedy of the commons}.

\section{Tragedy of the Commons}
\subsection{Intro}
In no situation, would anyone do the "right thing", where the right thing would be to maximize utility for everyone , because its not the same as the optimal for one particular person. Its like a \textbf{Prisoner's Dilemma}. 



\subsection{Review: prisoner dilemma}
Player A is rows, player B is columms

$
A  
\begin{array}{|c|c}
	B \\
    \hline
  -1,-1 & -9,0\\
  \hline
  0,-9 & -6,-6
\end{array}
$

If player A has to choose, he will choose the SECOND row, via \textbf{maximin} (-6 is bigger than -9). 
 

\subsection{???}
"Obamacare is the ultimate result of a tragedy of the commons". 

\subsection{PD -- read book/notes for more info!!}
always cooperate: $\frac{-1}{1-\gamma}$\\
always defect: $0 + (\frac{-6}{1-\gamma})\gamma$

so, which is better, always defect or always cooperate?if always cooperate is better, than what is the matehmatiacl equation that has to be true?

$\frac{-1}{1-\gamma} > \frac{-6 \gamma}{1-\gamma}$
$\Rightarrow \frac{-1}{-6} < \gamma$
$\Rightarrow \gamma > \frac{1}{6}$

\end{document}